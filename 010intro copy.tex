There has long been a disconnect between seismic risk assessment in earthquake engineering, which tends to focus on buildings, bridges, and occasionally, some narrow metric of impact, and urban planning, which has a long history of looking at the relative impacts of events and policy to different demographic groups and communities. For example, engineers typically use physical metrics such as the post-earthquake connectivity loss, which quantifies the decrease in the number of origins or generators connected to a destination node~\cite[e.g.,][]{duenas-osorio_seismic_2007}, or the post-earthquake travel distance between two locations of interest~\cite[e.g.,][]{chang_probabilistic_2000}. In contrast, one metric popular in urban planning is accessibility, which measures how easily people can get to desirable destinations~\cite{niemeier_accessibility:_1997}.

%Here's a fascinating question
Accessibility can be measured in many ways, including individual accessiblity, economic benefits of accessiblity, and mode-destination accessibility~\cite{geurs_accessibility_2004}. Mode-destination accessibility is particularly applicable to measuring the impacts of catastrophes, such as earthquakes, because while it is intuitive that certain destinations might be more critical for people in certain locations or from different market segments (such as low income or high income), the magnitude of these variations might not be as obvious. The mode-destination accessibility is computed by taking the sum of the log values of the utilities of each destination over all possible destinations and travel modes, where the utility decreases if getting to that destination is more costly or time-intensive~\cite{handy_measuring_1997}.

%Here's what people have tried to do (in brief: not a full related work section, but a high level assesment)
%\textcolor{red}{TODO: here is what people have tried to do. note how other researchers have identified at-risk areas in the bay area}
Mode-destination accessibility offers distinct advantages to other popular performance measures for identifying areas with high risk of losses from transportation network damage. The majority of prior work has focused on connectivity or flow on highly aggregated networks~\cite[e.g.,][]{lee_post-hazard_2011,xxx}, which do not directly capture the human impact, nor are necessarily representative of most of a region's trips, which likely have different origins and destinations.  Another popular option has been portfolio losses~\cite[e.g.,][]{padgett_regional_2010}, which capture direct economic losses due to bridge damage only. Recently, fixed-demand travel time increase has become a popular performance metric~\cite[e.g.,][]{kiremidjian_seismic_2007,jayaram_efficient_2010}, which offers some insight into the impact to people, in aggregate, in a region. However, travel time increase does not capture the fact that some destinations and trips have higher value than others, nor that demand may change after an earthquake. Thus, neither of these performance measures directly enable comparing risk between communities to identify ones at risk. In contrast, mode-destination accessibility directly captures the impacts to people considering the desirability or utility of the different destinations, and also is commonly computed separately for different communities or groups, which enables direct comparisons.
Paired with an appropriate model, accessibility can capture varying travel demand and some emergency response behaviors, such as switching to less-impacted travel routes and modes.

Regardless of the performance measure chosen, a reasonably representative transportation risk study suggests capturing the interdependencies with other transits systems and alternative travel modes, varying travel demand, and a relatively complete local road and highway network. While some recent work has captured the interdependencies between different infrastructure networks, such as electric power and water distribution~\cite{duenas-osorio_seismic_2007}, little work has considered the complexities within the transportation system itself. Furthermore, despite a few examples~\cite[e.g.,][]{kiremidjian_seismic_2007} of using a relatively complete road network, the majority of work has used rather simplified networks~\cite[e.g.,][]{lee_post-hazard_2011,jayaram_efficient_2010}, which may have an impact on the risk behavior. For example, not considering redundant local roads may lead to an overestimation of the risk, but not considering smaller bridges, such as pedestrian overpasses over key highways, may lead to underestimation of the risk. 
%here's the key challenge preventing further progress
Thus, the prior work suggests three questions not fully explored to date: 1) the risk of accessibility losses due to earthquakes, 2) the impact of interdependent transit systems and alternative travel modes---along with the road network---to the overall transportation system risk, and 3) infrastructure risk analysis of a large, reasonably comprehensive network in a probabilistic framework.

%Voila: here's our complete/partial/intermediate/awesome solution
Here we identify communities in a region at highest risk of losses in accessibility, as quantified by the mode-destination accessibility, using a probabilistic set of earthquake events, and a transportation model that captures transport mode choice and the interdependencies of the roads and transit systems.
%(additionally) and here's how it works.
We analyze the accessibility risk to geographic and demographic communities in the San Francisco Bay area (California) due to earthquake-related damages to the transportation network, which includes the road network, as well as transit networks and walking and biking options. First, we simulate a large set of earthquake scenarios, ground-motion intensity maps, and damage maps. Then, we use optimization to select a subset of the maps and analyze these maps with a high-fidelity, activity-based travel model. Finally, we analyze the predicted losses in accessibility according to 12 market segments used in practice for this region, based on a) income class,  and b) ratio of personal cars, trucks, vans, and other vehicles to workers in a household~\cite{org_mtc_2014}. In the subsequent discussion, we will use \emph{cars} to refer to any of these vehicle types. We find that location-to-location variations are more significant than variations across income classes. Furthermore, higher percentages of walking-based trips corresponds to higher resiliency to earthquake-related accessibility losses.
