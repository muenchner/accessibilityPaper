%So far we have focused on region-wide trends. However, we will now discuss the impacts at a local level for three communities, starting with one travel analysis zone (TAZ) of the downtown financial district of San Francisco (Figure~\ref{fig:equity_study_area}). As mentioned, San Francisco in general is expected to experience less loss in accessibility than most other communities. 
Two factors may explain this San Francisco TAZ's lower accessibility losses relative to most other communities.
%something about walking!
First, it differs dramatically from many other TAZs in having a small percentage of trips made by car (38\% versus an average of 85\% across all TAZs). Households traveling by foot or bike are less influenced by network damage, because foot travel routes and travel times are assumed to not be affected by bridge damage and road congestion. Additionally, trips by foot and bike tend to be to destinations that are shorter distances away than trips made via other modes. 
 % loss exceedance curve. something about they take short trips.
 Second, the times for trips to and from work are similar to that of other TAZs, and the average trip distance is only 7\% lower than the average for all trips region-wide. So the trip times and lengths do not explain the differences in accessibility losses in this TAZ.
%Third, as mentioned in the last section, areas away from the relatively more active Hayward and Calaveras faults in the East Bay have generally lower expected losses in accessibility. This is the case for this San Francisco TAZ. Specifically, the financial district is approximately 11 miles (18km) West of the closest segment of the Hayward Fault, with most areas even further away. Nonetheless, this is still close enough for some impact from these faults, e.g., as shown in Figure~\ref{fig:scen_acc}{(b)}. So, the results suggest that one factor to the relatively lower risk for San Francisco is being located on the San Francisco Peninsula (with a moderate separation distance from the East Bay faults), but not the key factor.
%%TODO: consider making a heat map of destinations from SF. Are they mostly avoiding East Bay bridges???
The data thus suggests that a major cause for the areas low accessibility risk is the low dependence on cars for mobility. 