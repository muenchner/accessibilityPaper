%While physical conditions contribute significantly to the risk of Pacifica, CA, trip length and bridge vulnerability play a big role in the accessibility risk for Danville, CA (Figure~\ref{fig:equity_study_area}). 

% We will first examine the trip length characteristics for Danville. %As illustrated in Figure~\ref{fig:time_distance_pdfs}, t
Pre-earthquake commute trips from Danville are on average longer in distance and duration than the previous communities. The average length of a trip from Danville is 85\% longer than the average over all trips in the study region. More specifically, there is a relatively higher proportion of trips taking 60-74 minutes and traveling over 40 miles than in the previous communities. These longer trips have more opportunities to be impacted by road closures, because more roads and bridges will be used to complete the trip. Moreover, the road layout near Danville mandates many highway trips, which increase the likelihood of crossing (damage prone) bridges. 

%\begin{figure}
%\centering
%\includegraphics[width=\textwidth]{../FIGS/equity_time_distance_Danville.eps} 
%\caption{Trip distributions for trips originating from Danville (TAZ 1161) for change over baseline after three earthquake events for a) trip length and b) trip time.}
%\label{fig:time_distance_loss_danville}
%\end{figure}

Bridge damage is important for many regions, including Danville, because the percentage of car-based trips is high (85\% of all trips from Danville). For all three simulations shown in Figure~\ref{fig:scen_acc}, some bridges in the Oakland area are damaged and thsu closed. In addition, in the first two simulations, there are closures to the north of Danville, which represents one of the two main travel routes from Danville. There are also scattered closed bridges to the west of Danville, a top travel corridor for people of Danville because of the workplace centers in San Francisco, Oakland, and Silicon Valley (all to the west). As for transit, in the first two events, all BART lines are closed, so there are limited alternatives to the popular road routes. The result is that the residents of Danville have reduced access to their normal destinations after all these events. 
Looking at the rate of bridge damage across all of the earthquake simulations in Figure~\ref{fig:scen_acc}{g}, we see that bridges in the Oakland-Berkeley area and to the north of Danville are particularly likely to be damaged. This suggests that Danville's proximity to  vulnerable bridges contributes to its accessibility risk.

%\begin{figure*}[t]
%    \centering
%    \begin{tabular}{cc}
%    \includegraphics[width=0.4\textwidth]{../FIGS/bridge_scen_154.eps} &
%    \includegraphics[width=0.4\textwidth]{../FIGS/bridge_scen_198.eps} \\
%     (a) Hayward, M7.05 & (b) San Andreas, M7.05 \\ 
%    \includegraphics[width=0.4\textwidth]{../FIGS/bridge_scen_196.eps} &
%        \includegraphics[width=0.4\textwidth]{../FIGS/equity_probDamageBig.eps} \\
%        (c) San Andreas, M8.25 & (d) Weighted average   
%    \end{tabular}
%\caption{Bridge damage state: a) after M7.05 Hayward, b) after M7.05 San Andreas, c) after M8.25 San Andreas, d) expected damage over all events (red is high likelihood  of extensive or complete bridge damage, and white is low likelihood).}
%\label{fig:bridge_ds}
%\end{figure*}


