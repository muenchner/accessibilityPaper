%While physical conditions contribute significantly to the risk of Pacifica, CA, trip length and bridge vulnerability play a big role in the accessibility risk for Danville, CA (Figure~\ref{fig:equity_study_area}). 

We will first examine the trip length characteristics for Danville. %As illustrated in Figure~\ref{fig:time_distance_pdfs}, t
The distribution of pre-earthquake commute trips from Danville is shifted towards both longer distance and longer time than the communities we have studied so far; for example, the average length of a trip from Danville is 85\% longer than the average over all trips originating from any TAZ. More specifically, there is a relatively higher proportion of trips taking 60-74 minutes and traveling over 40 miles than in the other communities. The consequence of these longer trips is more opportunities to be impacted by a road closure, simply because more roads and bridges will be used. Moreover, the road layout near Danville mandates many highway trips, which increase the likelihood of crossing bridges; bridges are the part of the network for which we model the vulnerability. 

%\begin{figure}
%\centering
%\includegraphics[width=\textwidth]{../FIGS/equity_time_distance_Danville.eps} 
%\caption{Trip distributions for trips originating from Danville (TAZ 1161) for change over baseline after three earthquake events for a) trip length and b) trip time.}
%\label{fig:time_distance_loss_danville}
%\end{figure}

Next, we look at patterns of expected bridge damage. Bridge damage is important for many regions, including Danville, because the percentage of car-based trips is high (85\% of all trips on average, and also 85\% of Danville-origin trips). For damage map realizations for the three  earthquake events we introduced---M6.85 Hayward Fault, M7.45 San Andreas Fault, M6.35 Great Valley Fault---some bridges in the Oakland area are in the extensive or greater damage state (Figure~\ref{fig:scen_acc}{(a,c,e)}). These correspond to bridge closures in the model. In addition, in the first two cases, there are closures to the north of Danville, which represents one of the two main travel routes from Danville. There are also scattered closed bridges to the west of Danville, a top travel corridor for people of Danville because of the workplace centers in San Francisco, Oakland, and Silicon Valley (all to the west). As for transit, in the first two events, all BART lines are closed, so there are limited alternatives to the popular road routes. The result is that the residents of Danville have reduced access to their normal destinations after all these events. 

We can also look at bridge damage in a probabilistic event-set-based manner. The expected damage over all events represents the annual rate of a bridge being in the extensive or complete damage state for an extensively-sampled, hazard-consistent set of damage maps (Figure~\ref{fig:scen_acc}{(g)}). This figure indicates that bridges in the Oakland-Berkeley area are particularly likely to be damaged. We also see a few high likelihood bridges to the North of Danville. Thus, the data suggests that the relative position of high-risk bridges to Danville contributes to this community's accessibility risk.

%\begin{figure*}[t]
%    \centering
%    \begin{tabular}{cc}
%    \includegraphics[width=0.4\textwidth]{../FIGS/bridge_scen_154.eps} &
%    \includegraphics[width=0.4\textwidth]{../FIGS/bridge_scen_198.eps} \\
%     (a) Hayward, M7.05 & (b) San Andreas, M7.05 \\ 
%    \includegraphics[width=0.4\textwidth]{../FIGS/bridge_scen_196.eps} &
%        \includegraphics[width=0.4\textwidth]{../FIGS/equity_probDamageBig.eps} \\
%        (c) San Andreas, M8.25 & (d) Weighted average   
%    \end{tabular}
%\caption{Bridge damage state: a) after M7.05 Hayward, b) after M7.05 San Andreas, c) after M8.25 San Andreas, d) expected damage over all events (red is high likelihood  of extensive or complete bridge damage, and white is low likelihood).}
%\label{fig:bridge_ds}
%\end{figure*}


